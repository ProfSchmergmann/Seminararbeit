%! Suppress = NonMatchingIf
% Schriftart und Sprache
\usepackage{lmodern}
\usepackage[T1]{fontenc}
\usepackage[utf8]{inputenc}
\usepackage[ngerman]{babel}

% Für KOMA-Script
\usepackage{scrhack}

% Andere wichtige Pakete
\usepackage{url}
\usepackage{pdfpages}
\usepackage{csquotes}
\usepackage{microtype}
\usepackage{amsfonts}
\usepackage{tabularx}
\usepackage{diagbox}
\usepackage{caption}
\captionsetup{position=above,labelfont=bf,textfont=it}

% Zeilenabstand zwischen Aufzählungen verringern
\usepackage{paralist}

\newcommand*\NewPage{\newpage\null\thispagestyle{empty}\newpage}

% Farben
\usepackage{xcolor}
\definecolor{gray}{rgb}{0.4,0.4,0.4}
\definecolor{maroon}{rgb}{0.5,0,0}
\definecolor{darkgreen}{rgb}{0,0.5,0}

% Bilder
\usepackage{graphicx}
\graphicspath{{include/img}}

% Tikz Graphiken
\usepackage{tikz}
\tikzset{every picture/.style={line width=0.75pt}} %set default line width to 0.75pt

% Paragraphen
\RedeclareSectionCommands[
	beforeskip=-3.25ex plus -1ex minus -0.2ex,
	runin=false,
	afterskip=-\parskip
]{paragraph,subparagraph}

% Einrückung verhindern
\setlength{\parindent}{0em}

% Listings
\usepackage{listings}

\lstset{
	backgroundcolor=\color{white},
	basicstyle=\ttfamily\footnotesize,
	breakatwhitespace=false,
	breaklines=true,
	captionpos=t,
	columns=fullflexible,
	commentstyle=\color{gray}\upshape,
	frame=single,
	numbers=left,
	numbersep=5pt,
	numberstyle=\tiny\color{gray},
	rulecolor=\color{black},
	showspaces=false,
	showstringspaces=false,
	showtabs=false,
	stepnumber=1,
	tabsize=2,
	xleftmargin=.02\textwidth,
	xrightmargin=.02\textwidth
}

\lstdefinelanguage{XML}{
	morestring=[b]",
	moredelim=[s][\bfseries\color{maroon}]{<}{\ },
	moredelim=[s][\bfseries\color{maroon}]{</}{>},
	moredelim=[l][\bfseries\color{maroon}]{/>},
	moredelim=[l][\bfseries\color{maroon}]{>},
	morecomment=[s]{<?}{?>},
	morecomment=[s]{<!--}{-->},
	commentstyle=\color{darkgreen},
	stringstyle=\color{blue},
	identifierstyle=\color{red}
}

% JavaScript
\lstdefinelanguage{JavaScript}{
	keywords={typeof, new, true, false, catch, function, return, null, catch, switch, var, if, in, while, do, else, case, break},
	keywordstyle=\color{blue}\bfseries,
	ndkeywords={class, export, boolean, throw, implements, import, this},
	ndkeywordstyle=\color{darkgray}\bfseries,
	identifierstyle=\color{black},
	sensitive=false,
	comment=[l]{//},
	morecomment=[s]{/*}{*/},
	commentstyle=\color{purple}\ttfamily,
	stringstyle=\color{red}\ttfamily,
	morestring=[b]',
	morestring=[b]"
}

% Titel, Autor, Datum
\newcommand*{\Titel}{
	Architekturen und Verfahren zur Absicherung des Informationsaustauschs im Client-Server Modell
}
\newcommand*{\Name}{Sven Bergmann}
\newcommand*{\Matrikelnummer}{3231105}
\title{\Titel}
\author{\Name}
\date{\today}

% Kopf- und Fusszeile
\usepackage[
	headsepline,
	footsepline,
	autooneside=false,
	automark
]{scrlayer-scrpage}

% Um die Platzhalter zu leeren
\clearpairofpagestyles

% head definieren
\automark[subsection]{section}

\ihead{\leftmark} % Kopfzeile innen
\chead{} % Kopfzeile mitte
\ohead{\Ifstr{\leftmark}{\rightbotmark}{}{\rightbotmark}} % Kopfzeile aussen
\ifoot{\Name, CAE GmbH Stolberg} % Fußzeile innen
\cfoot{} % Fußzeile mitte
\ofoot{\thepage} % Fußzeile aussen mit Seitenzahl

% Kapitelnummerierung in der Kopfzeile aus
\renewcommand*{\sectionmarkformat}{}

% FH Aachen ----------------------

% Seiteneinstellungen
\usepackage[
	a4paper,
	left=5cm,
	right=2cm,
	top=2cm,
	bottom=2cm
]{geometry}

% Zeilenabstand
\usepackage[onehalfspacing]{setspace}
\BeforeStartingTOC[toc]{\singlespacing}

% FH Aachen ----------------------

% BibLaTeX
\usepackage[
	maxbibnames=99,
	style=alphabetic
]{biblatex}
\addbibresource{include/Seminararbeit.bib}

% Zeilenabstand im Literatureintrag zurücksetzen
\AtBeginBibliography{\singlespacing}
% Zeilenabstand zwischen den einzelnen Einträgen im Literaturverzeichnis setzen
\setlength{\bibitemsep}{1.5\itemsep}

% Load hyperref after all other packages
\usepackage[
	bookmarks=true,
	bookmarksopen=false,
	breaklinks=false,
	colorlinks=true,
	pdfauthor={\Name},
	pdfborder={0 0 1},
	pdfsubject={Seminararbeit},
	pdftitle={\Titel},
	unicode=true
]{hyperref}

\usepackage{bookmark}

% Glossary
\usepackage[acronym,toc,nopostdot]{glossaries}
\makenoidxglossaries % use TeX to sort
% #1 - reference e.g. api
% #2 - Short e.g. API
% #3 - Full name e.g. Application Programming Interface
% #4 - Description
\newcommand{\newdefineabbreviation}[4]
{
% Glossary entry
	\newglossaryentry{#1_glossary}
	{text={#2},
		long={#3},
		name={\glsentrylong{#1_glossary} (\glsentrytext{#1_glossary})},
		description={#4}
	}

% Acronym
	\newglossaryentry{#1}
	{type=\acronymtype,
		name={\glsentrytext{#1_glossary}}, % Short
		description={\glsentrylong{#1_glossary}}, % Full name
		first={\glsentryname{#1_glossary}\glsadd{#1_glossary}},
		see=[Glossar:]{#1_glossary} % Reference to corresponding glossary entry
	}
}
\loadglsentries{include/glossary}
