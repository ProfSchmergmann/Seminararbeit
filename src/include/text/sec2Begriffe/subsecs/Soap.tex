\subsection{SOAP}\label{subsec:soap}
	Das \gls{soap}, zu Deutsch \enquote{einfaches Objektzugriffsprotokoll}, ist ein Protokoll,
	welches in einer dezentralen Umgebung für den Austausch von Informationen
	zwischen verschiedenen Systemen genutzt werden kann.
	Es wurde ursprünglich von Microsoft entwickelt, um alte Übertragungstechnologien,
	wie beispielsweise das \gls{dcom} oder die \gls{corba}, zu ersetzen.
	\gls{soap} wurde als sehr flexibles, erweiterbares,
	aber auch leichtgewichtiges Protokoll entwickelt,
	weshalb auch häufig Abkürzungen, wie zum Beispiel \gls{ws-security},
	\gls{ws-reliableMessaging} oder \gls{ws-addressing} damit assoziiert werden.
	Es umfasst gewisse Regeln,
	welche die zu verschickenden Informationen komplexer machen und \gls{overhead} produzieren,
	jedoch für einen gewissen Standard sorgen,
	welchen jede \gls{soap} \gls{api} versteht.
	Dieses Verfahren befindet sich seit 2007 in der Version 1.2
	und wurde als offizielles Protokoll von der \gls{w3c} definiert~\cite[Vgl.][]{SOAP_W3C}.

	Damit ein \gls{soap} Web Service funktionieren kann,
	muss dieser ein \gls{wsdl} Dokument bereitstellen.
	In diesem Dokument befinden sich alle Informationen über die angebotenen Methoden,
	die Parameter und die Rückgabewerte dieser und wie der Kontakt hergestellt werden kann.
	Dieses auf \gls{xml} basierende Dokument ist für jeden (zugelassenen) Nutzer einsehbar,
	womit alle Funktionen genutzt werden können.

	Die Zustellung einer \gls{soap} Nachricht kann mit dem Prinzip einer Hauspost verglichen werden.
	Der abgeschickte Brief kommt zuerst im Sekretariat (\gls{soap} Dispatcher) an,
	wo dieser dann geöffnet wird, um den richtigen Empfänger zu bestimmen.
	Die Nachricht wird dann an den eigentlichen Empfänger weitergeleitet,
	womit dieser dann darauf antworten kann.
	Der Client kann also nicht direkt mit dem Server kommunizieren,
	da alle Nachrichten immer vom Dispatcher gefiltert werden.
	Bei \gls{soap} in Kombination mit~\nameref{subsubsec:http} weist diese Nachricht
	\gls{http} POST als Methode auf und
	stellt diesen Request immer an dieselbe Adresse,
	was in Abbildung~\ref{fig:soapKommunikation} gezeigt ist.

	Listing~\ref{lst:SOAPListing} stellt ein Beispiel einer \gls{soap} Nachricht dar.
	Man sieht, dass hier drei Elemente definiert sind,
	welche im Folgenden näher erläutert werden.

	\subsubsection{Envelope}\label{subsubsec:soapEnvelope}
		Der Envelope funktioniert quasi als \enquote{Briefumschlag},
		welcher zwingend in dem zu übermittelnden Dokument angegeben sein muss.
		Das Element \textit{kann}, wie in Listing~\ref{lst:SOAPListing} gezeigt,
		ein Attribut zur Definition des Namensraums enthalten.
		Des Weiteren \textit{darf} es zusätzliche Attribute,
		wie zum Beispiel das Encoding, beinhalten,
		welche aber durch den vorhergehenden Namensraum qualifiziert sein müssen.
		Der~\nameref{subsubsec:soapBody} \textit{muss} als Kind Element vorhanden sein,
		jedoch \textit{kann} davor noch ein~\nameref{subsubsec:soapHeader} stehen.
		Alle weiteren Attribute \textit{müssen} nach dem~\nameref{subsubsec:soapBody} eingefügt werden.

		Da \gls{soap} keine traditionelle Versionierung bereitstellt,
		\textit{muss} ein Envelope den Namensraum \enquote{\url{http://schemas.xmlsoap.org/soap/envelope/}} referenzieren.
		Falls eine Nachricht mit einem anderen Namensraum, als dem des Programmes empfangen wird,
		muss die Applikation dies als \enquote{Version Error} behandeln und die Nachricht verwerfen.
		Wenn die Nachricht aber mit einem Frage/Antwort Protokoll (wie beispielsweise~\nameref{subsubsec:http}) geschickt wurde,
		muss auf die Nachricht mit einem \enquote{\gls{soap} VersionMismatch} Fehlercode geantwortet werden.

	\subsubsection{Header}\label{subsubsec:soapHeader}
		Falls der \gls{soap} Header existiert,
		\textit{muss} dieser das erste Kind Element des~\hyperref[subsubsec:soapEnvelope]{Envelopes} sein.
		Alle Kinder des Header Elements werden \enquote{Header Entries} genannt.
		Jedes dieser Elemente muss mit seinem Namensraum \gls{uri} und dem lokalen Namen voll identifizierbar sein,
		wobei der Namensraum immer gesetzt sein~\textit{muss}.
		Ansonsten existieren noch das~\nameref{p:soapEncodingStyle},
		~\nameref{p:soapMustUnderstand} und~\nameref{p:soapActor} Attribut,
		welche alle genutzt werden~\textit{können}.

		\paragraph{EncodingStyle}\label{p:soapEncodingStyle}
			Dieses Attribut definiert, mit welchen Regeln die Nachricht serialisiert wurde.
			Es \textit{kann} in jedem Element präsent sein und ist dann für alle Elemente innerhalb dieses Baumes verantwortlich,
			außer diese haben selbst jenes Attribut definiert.

			Der Wert stellt sich als eine geordnete Liste von einem oder mehreren \glspl{uri} dar,
			welche Regeln zur Deserialisierung beinhalten.
			Diese Liste ist nach absteigender Genauigkeit sortiert.

		\paragraph{Actor}\label{p:soapActor}
			Der \gls{soap} actor \textit{kann} genutzt werden,
			um den Empfänger eines~\nameref{subsubsec:soapHeader} Elements zu kennzeichnen.
			Dies ist zwingend notwendig damit die Nachricht effektiv sein kann,
			da die Nachrichten durch eine große Menge an \gls{soap} Vermittlern geleitet wird,
			welche diese entweder empfangen oder weiterleiten können.
			Daher sind nicht alle Teile sinnvoll oder bestimmt für das endgültige Ziel dieser Nachricht,
			sondern können weitergeleitet und auch verändert werden.
			Die Applikation, welche Attribute verändert hat,
			ist dann der neue \enquote{Partner} des finalen Empfängers.

		\paragraph{MustUnderstand}\label{p:soapMustUnderstand}
			Dieses Attribut \textit{kann} genutzt werden,
			um dem Leser zu signalisieren,
			ob dieser den Eintrag verpflichtend oder optional verarbeiten muss.
			Falls der Wert auf 1 steht,
			\textit{muss} der Empfänger entweder das Element korrekt nach den Richtlinien verarbeiten
			oder bei der Weiterverarbeitung der Nachricht fehlschlagen.
			Um die Effektivität der Nachrichtenverarbeitung gewährleisten zu können,
			muss dieses Attribut ebenfalls gesetzt sein.

	\subsubsection{Body}\label{subsubsec:soapBody}
		Der Body einer \gls{soap} Nachricht \textit{muss} direkt nach dem~\nameref{subsubsec:soapHeader} folgen,
		falls dieser präsent ist,
		andernfalls \textit{muss} er das erste Element sein.
		Dieses Element stellt einen simplen Mechanismus zur Informationsübertragung für den finalen Empfänger der Nachricht bereit.
		Äquivalent zum~\nameref{subsubsec:soapHeader} nennt man auch hier die Kind-Elemente \enquote{Body Entries},
		welche mit deren Namensraum \glspl{uri} und den lokalen Namen voll identifizierbar sein~\textit{müssen}.
		Des Weiteren \textit{kann} der~\nameref{p:soapEncodingStyle} definiert werden.