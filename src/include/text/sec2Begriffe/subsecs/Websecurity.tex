\subsection{Web-Security}\label{subsec:websecurity}
	Unter dem Begriff Web-Security versteht man im Allgemeinen die Absicherung der Übertragung von Dateien
	über das \gls{www} und damit auch in Verbindung mit dem~\nameref{subsubsec:http}.

	Da das~\nameref{subsubsec:http} grundsätzlich unverschlüsselt ist
	und daher jeder Mittelsmann die Kommunikation mithören und insbesondere auch verändern kann,
	muss \gls{https} genutzt werden.
	Das Protokoll wird auch~\enquote{\nameref{subsubsec:http} over \gls{tls}} genannt
	und liegt in der Schichtenarchitektur\footnote{s. Abbildung~\ref{fig:schichtenmodell}} auf der Anwendungsebene.
	Die Übertragung der Daten beim \gls{https} ist identisch zur Übertragung der Daten mit dem~\nameref{subsubsec:http}.
	Der Unterschied beider Protokolle liegt nur darin,
	dass bei~\gls{https} die Nutzdaten durch~\nameref{subsec:ssl} verschlüsselt wurden,
	jedoch IP-Adresse und Port weiterhin sichtbar sind.
	Ein wichtiger Begriff hierbei ist das sogenannte \enquote{X.509}~\cite[Vgl.][]{rfc5280} Zertifikat.
	Dieses folgt einer gewissen Struktur und beinhaltet unter anderem die Version,
	die Algorithmen-ID, den Aussteller, die Gültigkeit und den Inhaber.
	Mit solch einem Zertifikat kann ein Browser sicherstellen,
	dass tatsächlich mit der Internetseite/ dem Web Service kommuniziert wird.
	Die Zertifikate werden von einer \enquote{Certificate-Authority (CA)} ausgestellt,
	welche die Signatur übernimmt und diese in einen Verzeichnisdienst,
	welcher bereits ausgestellte Zertifikate enthält, einträgt.


