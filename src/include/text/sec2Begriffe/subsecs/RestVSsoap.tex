\subsection{Vergleich: REST vs.\ SOAP}\label{subsec:vergleich:-rest-vs.-soap}
	Tabelle~\ref{tab:vergleichSOAPREST} zeigt im Groben
	die Gemeinsamkeiten und Unterschiede der beiden Architekturen.
	Auffallend ist,
	dass beide ein \gls{uri} Adressmodell besitzen
	und über das~\nameref{subsubsec:http} Protokoll gesteuert bzw.\ angesprochen werden können.
	Das vereinfacht den Vergleich bezüglich der Performance enorm,
	welcher bewusst in der Tabelle fehlt,
	da verschiedene Gesichtspunkte unter Performance beleuchtet werden können.
	Wenn man aber nur den reinen Datenverkehr und damit zusammenhängend auch die Zeit nimmt,
	die eine Nachricht vom Clienten zum eigentlichen Endpunkt braucht,
	so ist~\nameref{subsec:rest} um einiges schneller.
	Dies liegt vor allem an dem \gls{overhead},
	welcher durch den~\nameref{subsec:soap} Dispatcher und
	die Anforderungen an eine~\nameref{subsec:soap} Nachricht entsteht.

	Des Weiteren wird die Schnittstellenbeschreibung nur vom~\nameref{subsec:soap} Standard gefordert,
	was sowohl positiv als auch negativ gesehen werden kann.
	Diese gibt nämlich Aufschluss darüber,
	welche Methoden und Services zur Nutzung bereitstehen.
	Jedoch wird auch wieder eine zusätzliche Anforderung an die Entwickler gestellt,
	selbst wenn das eventuell gar nicht benötigt wird.
	Auf der anderen Seite bietet eine~\hyperref[subsec:rest]{RESTful} \gls{api} eine generische Beschreibung,
	optimalerweise als Antwort auf jede Nachricht an den Server,
	welche durch das Prinzip der~\nameref{p:resthateoas} als verfügbare \glspl{uri} zurückgegeben wird.
	Die Schnittstellen sind daher auch unterschiedlich,
	da sich~\nameref{subsec:rest} strikt an die~\nameref{p:httpMethoden} hält,
	und~\nameref{subsec:soap} eben die Freiheit besitzt,
	die Schnittstellen anwendungsspezifisch zu definieren und
	in der Beschreibung des Web Services festhalten zu können.
	Eben hierbei zeichnet sich wieder die Leichtgewichtigkeit
	einer~\nameref{subsec:rest} \gls{api} dadurch ab,
	dass die vordefinierten Methoden genau so benutzt werden können
	und nichts zusätzlich implementiert werden muss.

	Zusammenfassend kann gesagt werden,
	dass es immer noch auf die persönlichen Vorlieben der Entwickler oder Firmen ankommt,
	welche Architektur tatsächlich genutzt werden soll.
	Für~\nameref{subsec:soap} gibt es zum Beispiel unzählige Bibliotheken,
	die die Implementierung eines solchen Services erheblich vereinfachen.
	Der Architekturstil~\nameref{subsec:rest} wird allerdings immer häufiger genutzt,
	da die Kopplung an die Funktionsweise des \gls{http} für viele einen großen Vorteil darstellt
	und dadurch auch mehr \glspl{framework} auf den Markt kommen.