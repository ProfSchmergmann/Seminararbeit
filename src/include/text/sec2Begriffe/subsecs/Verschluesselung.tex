\subsection{Verschlüsselung}\label{subsec:verschluesselung}
	Die Kryptographie, also Ver- und Entschlüsselungen von Informationen,
	hat eine sehr lange Geschichte und lässt sich bis ins alte Ägypten zurückführen.
	Konkret geht es mathematisch gesehen hierbei um eine Funktion oder auch Matrix,
	welche den \enquote{Klartext} in einen \enquote{Geheimtext} umwandelt.
	Dabei kann man die Zeichen des zu verschlüsselnden Textes entweder transponieren
	oder mit anderen Zeichen substituieren.
	Im besten Fall kann der \enquote{Geheimtext} dann nur mit dem Wissen über den Schlüssel
	wieder in den \enquote{Klartext} umgewandelt werden.
	Heutzutage unterscheidet man zwischen symmetrischen und asymmetrischen Verschlüsselungsverfahren,
	welche aber häufig zusammen eingesetzt werden,
	um die Rechenzeit zu verringern.
	Da asymmetrische Verschlüsselungsverfahren sicherer sind,
	aber viel mehr Zeit und Leistung benötigen,
	werden diese häufig nur zur Schlüsselübergabe verwendet und
	die Nutzdaten dann mit einem symmetrischen Verfahren verschlüsselt.

	\subsubsection{Symmetrische Verschlüsselung}\label{subsubsec:symVersch}
		Wie der Name schon sagt, erfolgen hierbei Chiffrierung und Dechiffrierung mit dem gleichen Schlüssel.
		Dieses Verfahren wird auch \enquote{Private-Key-Verfahren} genannt,
		da die Kommunikationspartner sicherstellen müssen, dass der gemeinsame Schlüssel privat,
		also geschützt vor öffentlichen Zugriffen, bleibt.
		Das erste bekannte Verfahren hierzu war die sogenannte \enquote{Caesar-Chiffre},
		bei welcher das Alphabet gegeneinander verschoben wird.
		Mathematisch ist die Verschlüsselung als $C_{K}(P) = (P + K) \bmod 26$ mit
		$P$ als Buchstabe des \enquote{Klartextes},
		$K$ als Anzahl der Verschiebung der Zeichen und
		$C$ als chiffrierter Buchstabe darstellbar.
		Die Entschlüsselung ist dann dementsprechend~$P_{K}(C) = (C - K) \bmod 26$.
		Hierbei erkennt man schon,
		dass sich ein solches Verfahren relativ einfach durchschauen lässt.
		Der nächste Schritt war also, nicht mehr nur eine Funktion zu verwenden,
		sondern ganze Algorithmen, welche mehrfach transponieren und substituieren.
		Der aktuelle symmetrische Verschlüsselungsstandard ist der \gls{aes},
		welcher im Oktober 2000 zertifiziert wurde und aus dem \gls{des} entstanden ist.
		Die Technologie beruht auf der Blockverschlüsselung,
		wobei einzelne Blöcke von 128, 192 oder 256 Bit verschlüsselt werden.
		Durch die Schlüssellänge wird \gls{aes} als sehr sicher bezeichnet und
		wird z.\ B.\ bei \gls{wpa2}, \gls{ssh} oder aber auch bei Komprimierungsverfahren wie \gls{7zip} verwendet.

	\subsubsection{Asymmetrische Verschlüsselung}\label{subsubsec:asymVersch}
		Die asymmetrische Verschlüsselung,
		auch \enquote{Public-Key-Verfahren} genannt,
		beruht auf der Verwendung von zwei statt nur einem Schlüssel.
		Dabei ist einer der beiden Schlüssel öffentlich,
		welcher zum Chiffrieren der Nachricht genutzt wird,
		der andere ist privat und wird zum Dechiffrieren der Nachricht genutzt.
		Dieses Verfahren funktioniert,
		da sogenannte Einwegfunktionen existieren,
		welche einfach zu berechnen aber schwierig umzukehren sind.
		Beispiele für solche Funktionen sind Modulo Operationen,
		Multiplizieren von Primzahlen oder Potenzieren.
		Für alle diese Funktionen gilt,
		dass $y = f(x)$ \enquote{einfach},
		also in akzeptabler Zeit zu berechnen ist,
		aber die Umkehrfunktion,
		also $f^{-1}(y) = x$ schwierig und
		nicht in akzeptabler Zeit ohne Vorkenntnis herausgefunden werden kann.
		Von diesem Verfahren existieren nicht so viele,
		wie bei den symmetrischen Chiffrierungen.
		Die bekanntesten sind der~\nameref{p:DiffieHellmanMerkleSchluesselaustausch} und
		die~\nameref{p:rsa} Verschlüsselung.
		Beide Verfahren profitieren davon,
		dass die oben erwähnten Einwegfunktionen auch mit \enquote{Falltür} existieren,
		was bedeutet, dass man die Umkehrfunktion durch Kenntnis einer Variablen leichter berechnen kann.

		\paragraph{Diffie-Hellman-Merkle Schlüsselaustausch}\label{p:DiffieHellmanMerkleSchluesselaustausch}
			Beim Diffie - Hellman - Merkle Schlüsselaustausch,
			häufig auch nur Diffie - Hellman genannt,
			wird angenommen, dass das Potenzieren einer Zahl einfach,
			jedoch das Logarithmieren schwierig ist.
			Angenommen es gibt zwei Partner, \textbf{A} und \textbf{B},
			welche sich auf zwei Zahlen verständigen,
			eine Primzahl $p$ und eine Basis~$g$.
			Des Weiteren generieren beide einen (zufälligen) privaten Schlüssel,
			$a$ für Partner \textbf{A} und $b$ für Partner~\textbf{B}.
			Diese Schlüssel werden dann genutzt,
			um den öffentlichen Schlüssel von \textbf{A} mit $\alpha = g^{a}\bmod p$
			und analog von \textbf{B} mit~$\beta = g^{b} \bmod p$ zu berechnen,
			welche dann ausgetauscht werden.
			Durch diesen Austausch und der Rechnung $\Psi = \beta^{a}\bmod p$,
			oder für~\textbf{B} $\Psi =\alpha^{b}\bmod p$ haben beide Partner den gemeinsamen Schlüssel~$\Psi$.

		\paragraph{RSA - Rivest, Shamir und Adleman}\label{p:rsa}
			Die RSA Verschlüsselung beruht darauf,
			dass die Primfaktorzerlegung großer Zahlen zu rechenaufwändig ist.
			Seien nun wieder zwei Partner \textbf{A} und \textbf{B} beteiligt,
			wobei dieses Mal \textbf{A} als Empfänger fungiert.
			\textbf{A} wählt die (möglichst großen) Primzahlen $p$ und $q$,
			berechnet daraus $n$ mit $n=p\cdot q$ und $\phi(n) = (p-1)\cdot (q-1)$
			mit der~\glslink{eulerphi}{Eulerschen $\phi$-Funktion}.
			Danach sucht \textbf{A} $e$ mit $ggT(e, \phi(n)) = 1$ und
			berechnet im Anschluss $d$ mit~$d\cdot e\bmod \phi(n) = 1$.
			Die beiden Zahlen $n$ und $e$ werden nun veröffentlicht und
			der Sender der Nachricht, in diesem Falle \textbf{B},
			kann diese jetzt mit $c = m^{e}\bmod n$ verschlüsseln.
			Die Daten werden von \textbf{A} wieder mit $m = c^{d}\bmod n$ entschlüsselt.