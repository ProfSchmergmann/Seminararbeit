\section*{Abstract}

	Diese Seminararbeit soll verschiedene Verfahren zur Absicherung des Informationsaustauschs
	im Client-Server Model vorstellen und vergleichen.
	Dafür werden zuerst alle relevanten Grundlagen geschaffen,
	indem auf die Funktionsweise des Informationsaustauschs über das Internet kurz,
	jedoch hinreichend,
	eingegangen wird.
	Auf Basis dessen werden Web Services definiert und bekannte Architekturen,
	wie SOAP und REST,
	vorgestellt, näher erläutert und verglichen.
	Danach wird die Verschlüsselung im Allgemeinen behandelt
	und die Brücke zum TLS-Protokoll geschlagen,
	was wiederum zur Definition des Begriffes Web-Security führt.
	\\
	Der darauf folgende Abschnitt wendet sich den eigentlichen Verfahren zu,
	wie der Zugriff über Authentifizierung und Autorisierung tatsächlich eingeschränkt oder gewährt werden kann.
	Das letzte Kapitel fügt alles zusammen
	und gibt einen Ausblick darüber,
	welches Verfahren für die Implementierung eines Web-Services tatsächlich genutzt werden könnte.
	Dies wird in Hinblick auf die Architektur,
	die persönliche Meinung,
	jedoch vor allem auch auf die gestellten Anforderungen,
	betrachtet.