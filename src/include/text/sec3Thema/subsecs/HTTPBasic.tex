\subsection{HTTP Authentication}\label{subsec:http-authentication}
	Die wohl einfachste Methode,
	eine \gls{authentifizierung} über das~\nameref{subsubsec:http} durchzuführen,
	ist es,
	ein schon vorhandenes Verfahren zu nutzen.
	Dieses ist in RFC 7235~\cite[Vgl.][]{rfc7235} spezifiziert und
	basiert auf einem einfachen \enquote{challenge-response} Modell,
	wobei die~\enquote{challenge} die Frage nach dem korrekten Passwort und
	die~\enquote{response} die Antwort mit dem korrekten Passwort darstellt.
	Der Client bekommt für jede Anfrage an einen Bereich,
	welcher eine \gls{autorisierung} benötigt,
	den Status-Code 401 (Unauthorized) vom Server zugeschickt.
	Zusätzlich muss der Server noch einige Header-Felder füllen und mitsenden,
	um dem Clienten unter anderem die
	\glslink{authentifizierung}{Authentifizierungsmethode} mitzuteilen.
	Das Feld \enquote{WWW-Authenticate} \textit{muss} zuerst die Methode
	und danach mindestens eine \enquote{challenge} enthalten,
	in welcher der Bereich spezifiziert ist.
	Um nicht immer wieder die Nutzerdaten
	\glslink{authentifizierung}{authentifizieren} zu müssen,
	gibt es noch das Feld \enquote{Authorization},
	welches in der Form \enquote{Authorization = credentials} die Informationen beinhaltet.
	Damit kann der Server dem Clienten Zugang zu weiteren Bereichen
	mit der gleichen \glslink{authentifizierung}{Authentifizierungsmethode} gewähren.
	Theoretisch gibt es noch zwei Proxy Felder,
	welche hier jedoch vernachlässigt werden.

	Im Folgenden werden ausgewählte Schemata aufgezeigt,
	die bei dieser Art der \gls{authentifizierung} genutzt werden können.

	\subsubsection{HTTP Basic Authentication}\label{subsubsec:http-basic-authentication}
		Die sogenannte \enquote{\gls{http} Basic Authentication}~\cite[Vgl.][]{rfc7617},
		setzt den Benutzernamen und das übergebene Passwort in der Form \enquote{BENUTZER:PW} zusammen,
		kodiert alles mit \gls{base64} und schickt es an den Server.
		Als \enquote{challenges} werden hierbei der Bereichsname,
		mit \enquote{realm=<Bereichsname>},
		und ein optionaler Parameter \enquote{charset},
		welcher das Kodierungsschema angibt,
		erwartet.
		Alle anderen Parameter müssen ignoriert werden.

		Diese Methode wird nur in Verbindung mit \gls{https} als sicher gewertet,
		da die Passwörter ansonsten quasi im Klartext mitgelesen werden können.

	\subsubsection{HTTP Digest Access Authentication}\label{subsubsec:http-digest-access-authentication}
		Die Antwort des Servers beinhaltet bei diesem Schema im \enquote{WWW-Authenticate} Header,
		neben Bereich und Namen,
		auch \enquote{qop}, \enquote{nonce} und~\enquote{opaque}.
		Der \enquote{Quality of Protection (qop)} Parameter \textit{muss} gesetzt sein und
		beschreibt die Sicherheitsqualität des Servers.
		Werte hierfür können \enquote{auth} für \gls{authentifizierung} und/oder
		\enquote{auth-int} für \gls{authentifizierung} mit Integritätsschutz sein.
		Ein einzigartiger, vom Server generierter \gls{string} wird \enquote{nonce} zugewiesen,
		der bei jeder \enquote{401} Response neu generiert wird.
		Die Inhalte dieser Zeichenfolge sind abhängig von der Implementierung,
		könnten aber beispielsweise etwas wie
		\enquote{timestamp : ETag : secret-data}~\cite[Vgl.][]{rfc7616} sein,
		was mit \gls{base64} kodiert wird.
		Der \enquote{opaque} Wert ist ebenfalls ein vom Server generierter
		\gls{string} (\gls{base64} oder hexadezimal),
		welcher einfach unverändert vom Clienten für den gleichen Bereich zurückgeschickt wird.
		Zur besseren Übersicht werden alle Werte,
		mit denen der Client antwortet,
		in Tabelle~\ref{tab:httpdigestclientresponse} dargestellt und erklärt.

		Falls der Server nun einen erfolgreichen Datenabgleich durchgeführt hat
		und der Client so \glslink{authentifizierung}{authentifiziert} werden konnte,
		wird ein \enquote{200 OK} Code mit der gewünschten Seite/
		dem gewünschten Inhalt zurückgeschickt.

	\subsubsection{NTLM HTTP Authentication}\label{subsubsec:ntlm-http-authentication}
		Das ursprünglich von Microsoft entwickelte NTLM Schema bietet eine weitere Anmeldemöglichkeit auf Servern,
		unter anderem mit dem \gls{singleSignOn} in Kombination mit einer Windows-Benutzeranmeldung.
		Bei diesem Schema wird als Antwort auf die Anfrage einer Resource
		der \enquote{WWW-Authenticate} Header mit dem Namen der \enquote{challenge} zurückgeschickt.
		Der Client schickt daraufhin im \enquote{Authorization} Header seinen Benutzernamen,
		welcher mit dem SPNEGO GSSAPI Mechanismus~\cite[Vgl.][]{rfc4178} generiert wurde.
		Diese Anfrage wird dann vom Server seinerseits dekodiert,
		mit einer Sicherheitsfunktion innerhalb des Protokolls überprüft,
		und falls der Kontext vollständig ist,
		wird die geforderte Resource mit dem erneut generierten~\glslink{hashfunktion}{Hash}
		des Mechanismus an den Clienten geschickt.
		Falls der Kontext nicht vollständig war,
		wiederholt sich dieser Prozess zwischen Client und Server so lange,
		bis der Server von der oben genannten Sicherheitsfunktion einen Wert zurückbekommt,
		der anzeigt,
		dass der Prozess erfolgreich war.
