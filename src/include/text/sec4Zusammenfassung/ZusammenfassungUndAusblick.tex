\section{Zusammenfassung und Ausblick}\label{sec:zusammenfassung-und-ausblick}
	\enquote{Einfach Einloggen und Herunterladen.}
	Für den einen oder anderen Leser der Seminararbeit mag das Wort \enquote{einfach}
	in diesem Satz mittlerweile ziemlich beschönigend aussehen.
	Sicherlich bekommen Anwender eher weniger davon mit,
	aber Anwendungsentwickler und Sicherheitsexperten müssen sich
	mit dem gar nicht so einfachen Thema dann doch auseinandersetzen.
	Zusammenfassend kann also gesagt werden,
	dass die Seminararbeit die theoretischen Grundlagen
	--- wenn auch nur oberflächlich ---
	einer Datenübertragung über das Internet aufzeigt
	und Basiswissen über~\hyperref[subsec:verschluesselung]{Kryptographie} bereitstellt.
	Damit wurden die beiden ersten Ziele,
	die Datenübertragung über das Internet im Ansatz zu verstehen und
	verschiedene Architekturen zur Realisierung dieser kennenzulernen,
	abgedeckt.

	Das zweite Kapitel handelte von einigen Verfahren zur Absicherung des Vorhergehenden
	und erfüllte damit das Ziel,
	die Wichtigkeit
	und Notwendigkeit der Absicherung eines Informationsaustauschs aufzufassen.
	Zuerst wurde die in \gls{http} integrierte \gls{authentifizierung} exemplarisch mit drei Varianten vorgestellt.
	Danach kamen kurz die~\nameref{subsec:api-keys},
	wobei bei beiden Verfahren darauf zu achten war,
	dass zwingend \gls{https} verwendet werden muss,
	um diese als sicher einstufen zu können.
	Das~\nameref{subsec:oauth},
	als relativ neues Protokoll,
	und~\nameref{subsec:openid-connect} als zusätzliche Schicht dazu,
	rundete das Kapitel ab.

	Für eine tatsächliche Implementierung eines~\nameref{subsec:web-services} mit beschränkten Zugriffsrechten
	und nur einer einmaligen Anmeldung eines Clienten würde sich
	eine~\hyperref[subsec:rest]{RESTful} \gls{api} anbieten,
	da diese um einiges leichtgewichtiger ist
	und momentan die bevorzugte Wahl großer Anbieter darstellt.

	Die Absicherung könnte dann entweder über eine Form der~\nameref{subsec:http-authentication},
	über die ähnlichen~\nameref{subsec:api-keys}
	oder eben über~\nameref{subsec:openid-connect} funktionieren.
	Welches Verfahren sich aber tatsächlich dazu eignet
	und vor allem den vorgegebenen Richtlinien entspricht,
	muss vor der Umsetzung noch einmal ausführlich diskutiert werden.
